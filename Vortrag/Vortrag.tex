\documentclass{beamer}
\usepackage[utf8]{inputenc}
\usepackage[ngerman]{babel}
\usepackage{amsmath}
\usepackage{amssymb}
\usepackage{graphicx}
%\usecolortheme{rose}
\setbeamertemplate{navigation symbols}{}
\setbeamertemplate{footline}[page number]
\title{Eine Analyse der 'Stop-and-Frisk' Politik des NYPD im Kontext des Racial Bias Vorwurfes}
\author{Arthur von der Heyden \\ und \\ Dagmar Lux}
\date{\today}
%Auto*innen der Studie und Jahr: Andrew Gelman, Jeffrey Fagan und Alex Kiss, 2007
\begin{document}
	\begin{frame}
		\titlepage
		\title{Eine Analyse der "Stop-and-Frisk" Politik des NYPD im Kontext des Vorwurfes des Racial Bias}
	\end{frame}
	\begin{frame}
		\frametitle{Agenda}
		\tableofcontents[hideallsubsections]
	\end{frame}
	%Dirty Zahlen, weil es mir einfach zu lange gedauert hat zu debuggen
	\section {1. Hintergrund und Daten}
		\subsection{Studiengegenstand}
		\subsection{Stop-and-Frisk}
		\subsection{Hintergrund: Racial Bias}
		\subsection{Aufbau des Datensatzes}
	\section{2. Modelle}
	\section{3. Resultate}
		\subsection{Ergebnisse}
		\subsection{Zusammenfassung}
	\section{4. Quellen}
	\section{5. Diskussion}
	\begin{frame}
		\frametitle{Studiengegenstand}
		\textbf{Analyse:} Anhalteraten von NewYorker*innen unterschiedlicher ethnischer Gruppen durch die Straßenpolizei.\newline
		\newline
		\textbf{Ziel:} Bewertung der zentralen Behauptung, dass rassenspezifische Anhalteraten nichts anderes widerspiegeln, als rassenspezifische Kriminalitätsraten.
	\end{frame}
	\begin{frame}
		%TODO: Definition highlighten
		\frametitle{Stop-and-Frisk}
		\begin{definition}
			Zivilpersonen auf der Straße vorübergehend \textbf{festhalten}, \textbf{befragen} \& manchmal auch \textbf{durchsuchen}
		\end{definition}
	\end{frame}
	\begin{frame}
		\frametitle{Hintergrund: Racial Bias}
		\begin{itemize}
			\item[\textbullet] \textbf{Späte 1990er:} Besorgnisäußerung über Schikane der Polizei gegenüber Minderheitengruppen
			\item[\textbullet] \textbf{2000:} Bundesbezirksgericht lässt Verwendung der Rasse als Durchsuchungskriterium zu, wenn eine ausdrückliche rassenspezifische Beschreibung der verdächtigten Person vorliegt
			\item[\textbullet] \textbf{2005}: Alpert, MacDonald und Dunham finden heraus, dass die Polizei eine Person aus einer Minderheit mit größerer Wahrscheinlichkeit als verdächtig einstuft, durch Berufung auf nicht-verhaltensbezogene Hinweise
		\end{itemize}
	\end{frame}
	\begin{frame}
		\frametitle{Aufbau des Datensatzes}
		\begin{itemize}
			\item[\textbullet] \textbf{UF-250-Formulare:} Aufzeichnungen über Kontrollen des NYPD
			\item[\textbullet] ca. 175.000 Kontrolldaten von Januar 1998 bis März 1999
			\item[\textbullet] Ausfüllung des Formulars nur unter bestimmten Bedingungen
			\item[\textbullet] Untersuchung der Formulare für Stichprobe von 5.000 Fällen \& 10.869 weitere Fälle, die 50\% der Fälle in einer nicht zufälligen Stichprobe von 8 der 75 Polizeibezirken repräsentieren
		\end{itemize}
	\end{frame}
	\begin{frame}
		\frametitle{Aufbau des Datensatzes}
		\begin{itemize}
			\item[\textbullet] \textbf{Ethnische Gruppen:}
			\begin{itemize}
				\item[\textbullet] Schwarze (Afroamerikaner*innen)
				%Eigentlich Latinx, aber ich moechte unsere Studierenden nicht verwirren
				\item[\textbullet] Hispanoamerikaner*innen (Latinas und Latinos)
				\item[\textbullet] Weiße (europäische Amerikaner*innen)
			\end{itemize}
		\end{itemize}
		\begin{itemize}
			\item[\textbullet] \textbf{Bezirke:}
			\begin{itemize}
				\item[\textbullet] $<$ 10\% Schwarze Bevölkerung
				\item[\textbullet] 10 – 40\% Schwarze Bevölkerung
				\item[\textbullet] $>$ 40\% Schwarze Bevölkerung
			\end{itemize}
		\end{itemize}
	\end{frame}
	\begin{frame}
		\frametitle{Modelle}
		Für jede ethnische Gruppe $ e=1, 2, 3$ und Bezirke $p=1, 2, 3$ wird die Anzahl der Stopps $y_{ep}$, mit Hilfe einer \\ \textbf{hierarchischen Poisson Regression} modelliert. \\
		$n_{ep}$ ist die Anzahl der jeweiligen Festnahmen.
		\newline
		\begin{align}& y_{ep} \sim Poisson \biggl(\frac{15}{12}n_{ep}e^{\mu+\alpha_e+\beta_p+\epsilon_{ep}}\biggl), \nonumber \\
			& \beta_p \sim N(0, \sigma_\beta^2), \quad
			\epsilon_{ep} \sim N(0, \sigma_\epsilon^2)  \nonumber
		\end{align}
		\newline
		Alternativ werden noch zwei weitere Spezifikationen angepasst:\\
		\begin{itemize}
			\item[\textbullet] Modellierung der Variabilität über die Bezirke
			\item[\textbullet] Modellierung des Verhältnisses von Stopps zu Festnahmen im Vorjahr
		\end{itemize}
	\end{frame}
	\begin{frame}
		%TODO: Pfad verallgemeinern 
		\frametitle{Ergebnisse}
		\begin{figure}[h]
			\centering
			\includegraphics[width=0.65\textwidth]{C:/Users/vonderheyden/Downloads/index.jpg}
			\caption{Geschätzte Rate mit der Personen in den verschiedenen Kategorien gestoppt wurden}
			\label{fig:Ergebnisse}
		\end{figure}
	\end{frame}
	\begin{frame}
		\frametitle{Ergebnisse}
		%Die wichtigsten Parameter sind die Raten der Stopps %$e^{\mu+\alpha_e}$ 
		%(im Vergleich zu den Festnahmen des Vorjahres)
		%für $e = 1, 2, 3$ \\
		%Die Kontrollen wegen Gewaltverbrechen und Waffendelikten machen mehr %als zwei Drittel der Stopps aus. \\
		% sie waren der umstrittenste  Aspekt bei der \glqq %Stopp-and-Frisk\grqq-Politik. \\ 
		\begin{table}
			\begin{tabular}{|l|l|l|}
				\hline
				& Schwarze & Hispanics \\
				\hline
				Gewaltverbrechen & 2.5x & 1.9x \\
				\hline
				Waffendelikte & 1.8x & 1.6x \\
				\hline
			\end{tabular}
		\end{table}
		\centering... häufiger als Weiße angehalten. \\
		% In den Kategorien Eigentums- und Drogendelikte wurden Weiße etwas häufiger angehalten. \\
		%Bei den beiden alternativ Modellen fallen die Ergebnisse ähnlich aus.
	\end{frame}
	\begin{frame}
		\frametitle{Zusammenfassung}
		\textbf{Ergebnisse:}
		\begin{itemize}
			\item[\textbullet] Schwarze und Hispanics werden häufiger gestoppt als Weiße:
			\begin{itemize}
				\item[\textbullet] \textbf{Stopps:} 51\% Schwarze, 33\% Hispanics 
				\item[\textbullet] \textbf{Bevölkerung in NY:} 26\% Schwarze, 24\% Hispanics
			\end{itemize}
			\item[\textbullet] Standards für Stopps bei Minderheiten lockerer, Häufigkeit gewollt und Zweckgebunden
		\end{itemize} 
			% \item Rassenstereotypisierung Minderwertigkeit Fremdartigkeit
	\end{frame}
	\begin{frame}
		\frametitle{Quellen}
		Gelman, A., Fagan, J., \& Kiss, A. (2007). An analysis of the New York City police department's 
		“stop-and-frisk” policy in the context of claims of racial bias. Journal of the American 
		statistical association, 102(479), 813-823. doi: 10.1198/016214506000001040
	\end{frame}
	\begin{frame}
		\frametitle{Diskussion}
		\begin{enumerate}
			\item<1> Glaubt ihr, dass auch in Deutschland Personen aus Minoritätsgruppen öfter untersucht werden als Weiße?
			\item<2> Aufgrund welcher Indizien würdet ihr als Zivilpolizist*in eine Person anhalten? 
		\end{enumerate}
	\end{frame}
\end{document}